% !TEX root = scombinatorics.tex
\documentclass[scombinatorics.tex]{subfiles}
\begin{document}

\chapter*{Notation}
\addcontentsline{toc}{chapter}{Preliminaries}

\label{minimax}


\def\medrel#1{\parbox[t]{6ex}{$\displaystyle\hfil #1$}}
\def\ceq#1#2#3{\parbox[t]{25ex}{$\displaystyle #1$}\medrel{#2}{$\displaystyle #3$}}

\thickmuskip=1mu plus 0.5mu minus 0.5mu
\medmuskip=0.5mu plus 0.2mu minus 0.2mu

Let $\U$ and $\V$ be two (large) sets.
Let $\phi(x\,;z)$ be a relation symbol, or a formula, whatever.
We denote by \emph{$\phi(\U\,;\V)$\/} the set $\{\<a,b\>\in\U\times\V\; :\; \phi(a\,;b)\}$ which we call: the relation defined by $\phi(x\,;z)$.
Sets of the form \emph{$\phi(\U\,;b)$}$=\{a\in\U\;:\; \phi(a\,;b)\}$, for some $b\in\V$, are called \emph{definable\/} sets.

In the first chapters we always restrict the study to the \emph{trace\/} of $\phi(\U\,;\V)$ on some finite set $A\times B$, where $A\subseteq\U$ and $B\subseteq\V$. 
We wite $\phi(A\,;B)$ for $\phi(\U\,;\V)\cap A\times B$.
Similarly, we write $\phi(A\,;b)$ for the trace of $\phi(\U\,;b)$ on $A$, that is, the set $\phi(\U\,;b)\cap A$.
We call it a definable subset of $A$.

We denote by \emph{$\phi(\U\,;b)_{b\in\V}$} and \emph{$\phi(A\,;b)_{b\in\V}$} the collection of definable sets, respectively definable subsets of $A$.

A similar notations is used with the roles of $\U$ and $\V$ interchanged.
I.e., we think of $\U$ as the set of parameters defining subsets of $\V$.
We write \emph{$\phi(x\,;z)^{\rm op}$\/} to signal that the formula $\phi(x\,;z)$ is considered in this \emph{dual\/} setting.

For the model theoretic minded, it would be more appropriate to call the definable sets \textit{global types}, respectively \textit{types over $A$}.
But this would make the terminology (if possible) more obscure.

For $k\le|A|$ we use following notation interchangeably
\smallskip

\ceq{\hfill\emph{$\displaystyle{A\choose k}$}\medrel{=}\emph{$\vphantom{\Big[}A^{(k)}$}}
{=}
{\Big\{A'\subseteq A\ :\ |A'|=k \Big\}}
\smallskip

For $n$ a non negative integer we write

\ceq{\hfill \emph{$(n]$}}
{=}
{\{1,\dots,n\},}
\bigskip

\ceq{\hfill \emph{$n$}\medrel{=}\emph{$[n)$}}
{=}
{\{0,\dots,n-1\},}
\bigskip

\ceq{\rm and\hfill\emph{$[n]$}}
{=}
{\{0,\dots,n\}}\noindent\nolinebreak[4]\hfill\rlap{\textcolor{red}{\Large\danger}}
\bigskip

Note that the latter conflicts with the notation used in combinatorics.

\end{document}
%\noindent\llap{\textcolor{red}{\Large\danger}\kern1.5ex}