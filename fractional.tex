% !TEX root = scombinatorics.tex
\documentclass[scombinatorics.tex]{subfiles}
\begin{document}
\chapter{}
\label{fractional}



\def\medrel#1{\parbox[t]{5ex}{$\displaystyle\hfil #1$}}
\def\ceq#1#2#3{\parbox[t]{20ex}{$\displaystyle #1$}\medrel{#2}{$\displaystyle #3$}}

\vskip-4ex
\emph{\LARGE\bf Work in progress}

\vskip10ex

%%%%%%%%%%%%%%%%%%%%%%%
%%%%%%%%%%%%%%%%%%%%%%%
%%%%%%%%%%%%%%%%%%%%%%%
%%%%%%%%%%%%%%%%%%%%%%%
%%%%%%%%%%%%%%%%%%%%%%%
%%%%%%%%%%%%%%%%%%%%%%%
\section{Transversals and packings}\label{Transversals_Packings}

Let $\phi(x\,;z)$ be given. 
Let $A\subseteq\U$ and $B\subseteq\V$ be finite sets. 

A subset $A'\subseteq A$ is a \emph{transversal\/} if $\phi(A',b)\neq\0$ for every $b\in B$. Equivalently, if the sets $\phi(a,B)_{a\in A'}$ cover $B$, i.e.

\ceq{\hfill\V}
{=}
{\bigcup_{a\in A'}\phi(a,B).}

The \emph{transversal number\/} is the smallest cardinality of a transversal $A'$.
It denoted by \emph{$\tau$.} 

A subset $B'\subseteq B$ is a \emph{packing\/} if $\phi(A,b)\cap\phi(A,b')=\0$ for every $b,b'\in B'$. Equivalently if $|\phi(a,B')|\le1$ for every $a\in A$.
The \emph{packing number\/} is the largest cardinality of a packing $B'$.
It is denoted by \emph{$\nu$.}

We may write \emph{$\tau_{\phi(A,B)}$} and  \emph{$\nu_{\phi(A,B)}$} when ambiguity is of concern. 

If $A'$ is a transversal and $B'\subseteq B$, then the sets $\phi(a,B')_{a\in A'}$ cover $B'$.
Now, suppose $B'$ is a packing, then these sets contain at most one element, hence $|B'|\le|A'|$.
Therefore, we always have $\nu\le\tau$.
Very little can be said in general about the reverse direction.

\begin{example}\label{expl_tr_pk}
  Let $\U=\RR^2$ and $\V$ is the set of lines in $\RR^2$.
  Let $\phi(x\,;z)$ be the incidence (that is, membership) relation.
  Let $A\subseteq\U$ and let $B\subseteq\V$ be a set of $n$ lines in generic position
  (any to lines intersect and every point is contained in at most two lines).
  Then $\tau=\lceil n/2\rceil$, as each point belongs to at most two lines, while $\nu=1$, as any two lines intersect.\QED
\end{example}  


A (fractional) multiset over $\U$ is a nonnegative real-valued function $A':\U\to\RR_{\ge0}$.
We interpret $A'(a)$ as the multiplicity of the element $a$ in the multiset $A'$.

We write $\phi(A',b)$ for the multiset $A'\cdot\Indicator_{\phi(\U,b)}$.

The \emph{support\/} of a multiset is the set where it is positive.
For multisets of finite support we define the \emph{size\/} to be\smallskip 

\ceq{\hfill\emph{$|A'|$}}
{=}
{\sum_{a\in\U}A'(a)}

A multiset over $\V$ and $\phi(a,B')$ are defined analogously.
 
If $A'$ and $A''$ are multisets we write $A'\le A''$ if $A'(a)\le A''(a)$ holds for every $a$.

A \emph{fractional transversal\/} is a multiset $A'\le\Indicator_A$ such that $|\phi(A', b)|\ge1$ for every $b\in B$.
The \emph{fractional transversal number\/} of $\phi(x\,;z)$, denoted by $\tau^*$, is the  infimum of the size of the fractional transversals of $\phi(x\,;z)$.

A fractional multiset $B$ over $\V$ is a \emph{fractional packing\/} if $|\phi(a,B)|\le1$ for every $a\in\U$.
The \emph{fractional packing number\/} of $\phi(x\,;z)$, denoted by $\nu^*$, is the supremum of the size of the fractional packings of $\phi(x\,;z)$.

The notation $\tau^*$ and $\nu^*$ is standard. 
It is not related with the dual/opposite of a formula, $\phi(x\,;z)^*$. 

\begin{example}
  The sets $\U,\V$ and the relation $\phi(A\,;B)$ are as in the example~\ref{expl_tr_pk}.
  Let $B'$ be a multiset that assigns $1/2$ to every line in $B$. Then $|\phi(a,B')|\le1$ holds because each point is contained in at most two lines.
  Then $\nu^*\ge |B'|=n/2$.
  It is easy to see that $\tau^*\ge n/2$.
  We claim that $\tau\le n/2$.
  If $n$ is even, use the same transversal $A'$ as in Example~\ref{expl_tr_pk} is even.
  If $n$ is odd, take $3$ any lines, and assign $1/2$ to the three intersection points.
  Proceed as in the even case with the other lines.\QED
\end{example}

\begin{exercise}
Let $\U=\RR$ and $\V$ is a set of finitely many closed intervals.
Let $\phi(x\,;z)$ be the membership relation.
Then $\nu=\tau$.
Hint: use induction on $\nu$.\QED
\end{exercise}

\begin{theorem}\label{thm_fractional_nu=tau}
 For all $\phi(x\,;z)$ and all finite sets $A\subseteq\U$ and $B\subseteq\V$, we have $\nu^*=\tau^*$ and this value is rational.
\end{theorem}
\begin{proof}
Let $A=\{a_1,\dots,a_m\}$ and $B=\{b_1,\dots,b_n\}$.
Let $F$ be the $\{0,1\}$-valued incidence matrix $\phi(a_i,b_j)$.
A multi-set over $\U$ is a naturally associated to a vector $0\le x\in\RR^m$.
A multi-set over $\U$ is associated to a vector $0\le y\in\RR^n$.
Then it is easy to verify that

\ceq{\hfill\tau^*}{=}{\min\,\big\{\,1_n^{\rm T}\;x\ :\ \ F\,x\ge 1_m,\ 0\le x\};}

\ceq{\hfill\nu^*}{=}{\max\big\{\,1_n^{\rm T}\ y\,\  :\ F^{\rm T}y\le 1_n,\,\  0\le y\}.}

Therefore, by he duality of linear programming $\nu^*=\tau^*$.

As $\tau^*$ is the minimum of the linear function $x\mapsto 1_n^{\rm T}\;x$ over a polyhedron, such minimum is attained at vertex.
The inequalities describing the polyhedron have rational coefficients, so also the vertices have rational coordinates.
\end{proof}

When $\phi(x\,;z)$ has finite \vc-dimension, we can bound the transversal number $\tau$ with a function of $\tau^*$. 

\begin{proposition}\label{prop_bound_fractional_trans}
  Let $\phi(x\,;z)$ have \vc-dimension $k$.
  Then for all finite sets $A\subseteq\U$ and $B\subseteq\V$
  
  \ceq{\hfill\tau}{\le}{ c\,k\,(\tau^*)^2\,\ln\tau^*}
  
  where $c$ is an absolute constant.
\end{proposition}
  
\begin{proof}
  Let $A'$ be an optimal fractional transversal.
  After normalizing, $A'$ defines a probability measure $\Pr$ on $\U$.
  Namely, $\Pr(\{a\})=A(a)/\tau^*$ for $a\in\U$.
  By the definition of fractional transversal, every set $\phi(A\,;b)_{b\in B}$ has measure at last $1/\tau^*$.
  By Proposition~\ref{prop_vc_sample}, for every $\epsilon>0$ there is a sample $s$ of size
  
  \ceq{\hfill n}
  {\le}
  {c\,\frac{k}{\epsilon^2}\log\frac{k}{\epsilon^2}}

  If we set $\epsilon=1/\tau^*$, then $\range(s)$ is a transversal and we obtain the required bound.
\end{proof}

The bound in the proposition above can be improved.
One can replace $(\tau^*)^2$ with  $\tau^*$ at the cost of a more difficult proof.

\section{Helly-type properties}

We now investigate methods of bounding $\tau^*=\nu^*$.
As motivation we cite a classical theorem of Helly.

\begin{proposition}[(Helly Theorem)]
Let $\Phi$ be a finite family of convex sets in $\RR^d$.
Assume that any $d+1$ sets from $\Phi$ have non-empty intersection.
Then the whole family $\Phi$ has non-empty intersection.\QED
\end{proposition}

Note that Helly's theorem does not hold for families of finite $\vc$-dimension.
A counter example of \vc-dimension $2$ can be constructed with a family containing sets that are unions of two finite intervals of the real line.

We will deal with the following property, which is more robust.

\begin{definition}
We say that $\phi(x\,;z)$ \emph{has fractional Helly number $k$\/} if for every $\alpha>0$ there is a $\beta>0$ such that:

if $B\subseteq\V$ is any finite set (say of cardinality $n$) such that

\ceq{\hfill \bigcap_{b\in B'}\phi(\U,b)}{\neq}{\0}\hfill for at least \ $\displaystyle\alpha{n\choose k}$ \ sets \ $\displaystyle B'\in{B\choose k}$

then

\ceq{\hfill\bigcap_{b\in B''}\phi(\U,b)}{\neq}{\0}\hfill for some $B''\subseteq B$ of cardinality $\ge\beta\,n$.

We say that $\phi$ has the \emph{fractional Helly property\/} if it has some finite Helly number.
The \emph{fractional Helly number\/} of $\phi$ is the smallest number $k$ satisfying the property above.\QED
\end{definition}

For future reference we remark that the $\beta$ above may be used to obtain a bound to $\nu$. In fact, if at least $\beta\,n$ definable sets intersect, then $\nu\le n-\beta\,n+1$.

\begin{theorem}[(Matou\v{s}ek)]
Let  $\phi(x\,;z)$ have dual \vc-dimension $<k$.
Then $\phi(x\,;z)$ has fractional Helly number $\le k$.
\end{theorem}

\begin{proof}
Let $\alpha$ be arbitrary and set $\beta=1/2m$ where $m$ is such that 

\ceq{\hfill \pi^*(m)}{<}{\frac\alpha4{m\choose k}.}

Assume for a contradiction that finite $B\subseteq\V$ contradicts the definition above.
That is,
 
\ceq{\hfill \bigcap_{b\in B'}\phi(\U,b)}{\neq}{\0}\hfill for at least \ $\displaystyle\alpha{n\choose k}$ \ sets \ $\displaystyle B'\in{B\choose k}$

and

\ceq{\hfill\bigcap_{b\in B''}\phi(\U,b)}{=}{\0}\hfill for all $B''\subseteq B$ of cardinality $\ge\beta\,n$.

Note that we can assume that $n>2m$ because for $n\le2m$ we can take any $B''$ of cardinality $1$.

We will find a set $B''\subseteq B$ of cardinality $m$ with many $\phi(x\,;z)^*$-definable subsets, more than $\pi^*(m)$, a contradiction.

Let $P$ be the set of pairs $B'\subseteq B''\subseteq B$ such that $|B'|=k$ and $|B''|=m$.
We say that a pair $B'\subseteq B''$ in $P$ is \textit{good\/} if there is $a\in\U$ such that $B'=\phi(a,B'')$.
That is, $B'$ is a $\phi(x\,;z)^*$-definable subset of $B''$.

\textit{Claim.} Assume the uniform probability on $P$.
Then the probability that a random pair is good is $\ge\alpha/4$.

We assume the claim for now and continue with the proof.
We can think that the random pair in $P$ is chosen by first picking $B''\in{B\choose m}$ with the uniform distribution and than $B'\in{B''\choose k\phantom{'}}$ again with the uniform distribution. 
(To put it more pedantically, we are applying the theorem of total probability.)
If the the probability that a pair is good is $\ge\alpha/4$, then for at least one $B''\in{B\choose m}$ the probability of finding a good subset $B'$ is $\ge\alpha/4$.
Therefore, $B''$ has $\ge\frac\alpha4{m\choose k}$ good subsets.
A contradiction which proves the theorem given the claim.

We now prove the claim.
There is another equivalent way to pick a random pair in $P$.
First we choose at random $B'\subseteq B$ of cardinality $k$, then we choose at random $m-k$ elements from $B\sm B'$.
By assumption, the probability that $B''$ is such that $\bigcap_{i\in B''}\phi(\U,b)\neq\0$ is at least $\alpha$.
So, assume that $B'$ is such, and fix any $a\in \bigcap_{b\in B'}\phi(\U,b)$.
By the assumption that we aim to contradict, there are $<\beta n$ many $b\in B$ such that $\phi(a,b)$.
Then the probability that all $b\in B''\sm B'$ are such that $\neg\phi(a,b)$ is at least

\ceq{\hfill\frac{\displaystyle{n-\beta n\choose m-k}}{\displaystyle{n-k\choose m-k}}}{\ge}{\prod^{m-k-1}_{i=0}\frac{n-\beta n-i}{n-k-i}}
\medrel{\ge}
$\displaystyle\prod^{m-k}_{i=0}\frac{n-\beta n-i}{n-k-i}$

\ceq{}{\ge}{\prod^{m-k}_{i=0}\frac{n-\beta n-m}{n-m}}
\medrel{\ge}
$\displaystyle\bigg(\frac{n-\beta n-m}{n-m}\bigg)^m$

\ceq{}{=}
{\bigg(1-\frac{\beta n}{n-m}\bigg)^m}

\smallskip recall that $n>2m$ and $\beta\le1/2m$, then

\ceq{}{\ge}{\displaystyle(1-2\beta)^m}
\medrel{=}
$\displaystyle\bigg(1-\frac{1}{m}\bigg)^m$
\medrel{\ge}
$\displaystyle\frac{1}{4}$

Therefore the probability that a random pair in $P$ is good is at least $\alpha/4$.
This proves the claim and with it the theorem.
\end{proof}


%%%%%%%%%%%%%%%%%%%%%%%%%%%
%%%%%%%%%%%%%%%%%%%%%%%%%%%
%%%%%%%%%%%%%%%%%%%%%%%%%%%
%%%%%%%%%%%%%%%%%%%%%%%%%%%
\section{The (p,q)-theorem}

For integers $p\ge q$ we say that $\phi(x\,;z)$ has the \emph{$(p,q)$-property\/} if out of any $p$ definable sets there are $q$ sets with non-empty intersection.

For every finite $A\subseteq\U$ and $B\subseteq\V$, and every $B'\subseteq B$ of cardinality $p$ such that there is some $B''\subseteq B'$ of cardinality $q$ such that

\ceq{\hfill \bigcap_{b\in B''}\phi(A\,;b)}
{\neq}
{\0}

For {\nip} formulas the $(p,q)$-property has this interesting consequence.
There is a bound for the cardinality of the transversal number $\tau$ that only depends on $p,q$ and the \vc-dimension of the formula.

\begin{theorem}[(Alon, Kleitman + Matou\v sek)]
  Let $p\ge q\ge k+1$ be natural numbers.
  There is a number $N=N(k,p,q)$ such that 
  $\tau\le N$ for all formulas $\phi(x\,;z)$ with the $(p,q)$-property and dual \vc-dimension $\le k$.
\end{theorem}

\begin{proof}
As we are not trying to optimize $N$, we may prove the theorem for $q=d+1$.
By Proposition~\ref{prop_bound_fractional_trans}, the transversal number is bounded by a function of $\tau^*$, so it is enough to bound $\tau^*$. By Proposition~\ref{thm_fractional_nu=tau}, we can equivalently bound the fractional packing number $\nu^*$, because it coincides with $\tau^*$.

Let $B'$ be an optimal fractional packing. By Theorem~\ref{thm_fractional_nu=tau} we may assume that $B'$ is rational valued. Therefore $B'=(1/m)\,C$ where $m$ is a positive integer and $C$ is a integral valued multiset over $\V$.

Replace $\V$ with $\V\times[m]$ and $B$ with $B\times[m]$.
Let $\phi_m(a\,;b,i)$ be equivalent to $\phi(a\,;b)$.
Now we can assume that $C\subseteq B\times[m]$ and that $|\phi_m(a,C)|\le m$ for every $a\in\U$.

\smallskip
Claim~1. $\phi_m(x\,;z,y)$ satisfies the $(p',q)$-property with $p'=qp$.

If a $p'$-collection of $\phi_m(A\,;z,y)$ definable sets contains $q$ copies of the same definable set, this is the required a collection with nonempty intersection.
So, suppose not.
Then it contains $p$ distinct $\phi$-definable sets.

\smallskip
Claim 2.
There is an $\alpha=\alpha(p,q)>0$ such that for every finite $D'\subseteq\V\times[m]$

\ceq{\hfill \bigcap_{d\in D'}\phi(\U,d)}{\neq}{\0}\hfill for at least \ $\displaystyle\alpha{n\choose q}$ \ sets \ $\displaystyle I\in{[n]\choose q}$.

By Claim~1, every $p'$-collection of $\phi_m$-definable sets contains at least one $q$-collection with non-empty intersection.
Every $q$-collection is contained in ${n-q\choose p'-q}$ many $p'$-collections.
Therefore the number $q$-collections of definable sets with non-empty intersection is

\ceq{\hfill\frac{\displaystyle{n\choose p'}}{\displaystyle{n-q\choose p'-q}}}{=}{\frac{\displaystyle{n\choose q}}{\displaystyle{p'\choose q}}.}

Therefore, claim~2 holds with \smash{$\displaystyle\alpha={p'\choose q}^{-1}$\kern-1ex}.

\medskip
\textit{Claim 3.}
There is a $\beta=\beta(p,q)>0$ such that for every $n$ and every $b_1,\dots,b_n\in\V'$ there is an $a\in\U$ such that $\phi(a,b_i)$ holds for at least $\beta n$ many $b_i$.

As $\phi'$ has the same \vc-codensity as $\phi$,  Caim~3 follows immediately from Claim~2 by Helly fractional theorem.

Now, apply Claim~3 with the elements of $C$ as $b_1,\dots,b_n$. Then $\nu^*=n/m$. So, from $\beta\,n\le |\phi(a,C)|\le m$, we obtain  $\nu^*\le 1/\beta$. So $N=1/\beta$ is the bound claimed by the theorem.
\end{proof}

\end{document}
