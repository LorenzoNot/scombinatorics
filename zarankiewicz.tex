% !TEX root = scombinatorics.tex
\documentclass[scombinatorics.tex]{subfiles}
\begin{document}

\chapter{Zarankiewicz problem(s)}

\label{zarankiewicz}

% \sections{The Beineke-Schwenk problem}

% Determine the quadruples $m,n;s,t$ of positive integers such that for every $\phi(x\,;z)$ and every pair $A\subseteq\U$ and $B\subseteq\V$ of cardinality $m$ respectively $n$ there are $A'\subseteq A$ and $B'\subseteq B$ such that either $A'\times B'\subseteq\phi(x\,;z)$ or $A'\times B'\subseteq\neq\phi(x\,;z)$.

% Sometimes the problem is formulated as follows.
% Given $s,t$ determine the least $n$ such that $n,n;s,t$ is a quadruple as above.


\def\medrel#1{\parbox[t]{6ex}{$\displaystyle\hfil #1$}}
\def\ceq#1#2#3{\parbox[t]{25ex}{$\displaystyle #1$}\medrel{#2}{$\displaystyle #3$}}

Let us start with presenting Zarankiewicz problem.
Let $A\subseteq\U$ and $B\subseteq\V$ have cardinality $m$ respectively $n$.
Given two integers $s,t$ (both at least $2$ to avoid trivialities), what is the maximal cardinality of a graph $\phi(A\,;B)$ that does not contain $A'\times B'$  for any $A'\subseteq A$ and $B'\subseteq B$ of cardinality $s$ respectively $t$?
Denote this maximum by \emph{$z(m, n; s, t)$.}
Define also \emph{$z(n; t)$} $=z(m, n; s, t)$.
In 1951 Zarankiewicz posed the problem of determining $z(n; 3)$ for $n = 4, 5, 6$ and the  general problem has also become known as the problem of Zarankiewicz.


\section{The K\H{o}v\'ari-S\'os-Tur\'an Theorem}
The theorem in this section is a classical result of K\H{o}v\'ari, S\'os, and Tur\'an.
They actually proved it for $z(n;t)$ but the proof easily generalizes.

In the proof we need the following generalization of the binomial coefficient. 
For any positive integer $t$ and any real $x$ we define

\ceq{\hfill\binom{x}{t}}{=}{\frac{x(x-1)\cdots(x-t+1)}{t!}.}

It is easy to verify that for any fixed $t$, this is a convex and strictly increasing function of $x$.

\begin{void_thm}[K\H{o}v\'ari-S\'os-Tur\'an Theorem]\label{thm_KST}
  For all $2\le s\le m$ and $2\le t\le n$
  \smallskip

  \ceq{\hfill z(m, n; s, t)}{<}{(s-1)^{1/t}(n-t+1)m^{1-1/t}+(t-1)m}

  or also

  \ceq{}{<}{c\,\big(n\,m^{1-1/t}+m\big)}
  
  for some $c = c(s,t)$.

  
\end{void_thm}
\begin{proof}
  Let $A\subseteq\U$ and $B\subseteq\V$ have cardinality $m$ respectively $n$.
  Let $P$ be the set of pairs $\<a,B'\>$ such that $B'\subseteq\phi(a\,;B)$ and $B'$ has cardinality $t$.
  Writing $d_a$ for the cardinality of $\phi(a\,;B)$, we have

  \ceq{\hfill|P|}{=}{\sum_{a\in A}\binom{d_a}{t}}

  Write $z$ for $|\phi(A\,;B)|$, and note that

  \ceq{\hfill z}{=}{\sum_{a\in A}d_a.}

  As \smash{$\displaystyle\binom{x}{t}$} is a convex function of $x$,
  \smallskip

  \ceq{(1)\hfill\binom{z/m}{t}}{\le}{\frac1m\,|P|.}
  \smallskip

  Now, suppose that $z=z(m, n; s, t)$.
  Then for any fixed $B'\subseteq B$ there are at most $s-1$ pairs $\<a,B'\>\in P$. 
  Hence

  \ceq{(2)\hfill|P|}{\le}{(s-1)\binom{n}{t}.}

  Together, (1) and (2) yield
  \smallskip

  \ceq{\hfill m\binom{z/m}{t}}{\le}{(s-1)\binom{n}{t}.}
  \smallskip

  As \smash{$\displaystyle\binom{x}{t}$} is strictly increasing function of $x$, and $z/m<n$,
  \smallskip

  \ceq{\hfill m \Big(\frac{z}{m}-t+1\Big)^t}{<}{(s-1)\big(n-t+1\big)^t}
  \medskip

  which easily yields the theorem.
\end{proof}


\section{The {\sc nip} case}


\end{document}